%%%%%%%%%%%%%%%%%%%%%%%%%%%%%%%%%%%%%%%%%
% Medium Length Professional CV
% LaTeX Template
% Version 2.0 (8/5/13)
%
% This template has been downloaded from:
% http://www.LaTeXTemplates.com
%
% Original author:
% Trey Hunner (http://www.treyhunner.com/)
%
% Important note:
% This template requires the resume.cls file to be in the same directory as the
% .tex file. The resume.cls file provides the resume style used for structuring the
% document.
%
%%%%%%%%%%%%%%%%%%%%%%%%%%%%%%%%%%%%%%%%%

%----------------------------------------------------------------------------------------
%	PACKAGES AND OTHER DOCUMENT CONFIGURATIONS
%----------------------------------------------------------------------------------------

\documentclass{resume} % Use the custom resume.cls style

\usepackage{scrextend}
\usepackage[utf8]{inputenc}
\usepackage[hidelinks]{hyperref}
\usepackage[left=0.5in,top=0.3in,right=0.5in,bottom=0.3in]{geometry} % Document margins
\usepackage{hyperref}

\hypersetup{urlcolor=magenta}

\usepackage{helvet}
\renewcommand{\familydefault}{\sfdefault}
 
\name{Javier Ribera} % Your name
\address{\href{mailto:javier@ribera.me}{javier@ribera.me}}
%\address{\url{http://ribera.me}}
\begin{document}

% %----------------------------------------------------------------------------------------
% %	OBJECTIVE SECTION
% %----------------------------------------------------------------------------------------
% \vspace{-8pt}
% \begin{rSection}{Objective}
% To obtain a full-time position in Deep Learning applied to Computer Vision/Image Processing, starting January 2019.
% 
% \end{rSection}

%----------------------------------------------------------------------------------------
%	WORK EXPERIENCE SECTION
%----------------------------------------------------------------------------------------


\vspace{-10pt}
\begin{rSection}{Experience}

\begin{rSubsection}{Co-founder}{8/2020 - Present}{\textbf{$\hookrightarrow$ Oqullo}}{Miami, FL}
\setlength{\itemindent}{.3in}
\item[-] Built from the ground-up a platform for foot traffic analytics using existing camera networks. 
\item[-] Led a team of developers, designed roadmap, planned budget, and coordinated labeling efforts and sales prospecting.
\item[-] Developed a Pedestrian Tracking System across multiple cameras. Research and implementation on-premise and in AWS.
\item[-] Technologies used: CNN, TensorRT, Person-ReID, Docker, React
\end{rSubsection}

\begin{rSubsection}{Sr. Algorithm Engineer}{Jan 2019 - Jul 2020}{\textbf{$\hookrightarrow$ Samsung}}{San Jose, CA}
\setlength{\itemindent}{.3in}
\item[-] Researched and delivered Machine/Deep Learning models to evaluate the image quality of Samsung's new displays.
\item[-] Patented an objective visual quality evaluation of Samsung's display pipeline.
\item[-] Evaluated image quality, visualy and quantitatively, with Python and Matlab simulations of displays and the Human Visual System.
\item[-] Technologies used: PyTorch, CNN, FCN, MLP, ELM, PSNR, SSIM, SCIELAB, ISETBIO
\end{rSubsection}


\begin{rSubsection}{Research Assistant}{Jan 2016 - Dec 2018}{\textbf{$\hookrightarrow$ (ARPA-E Project)} Video and Image Processing (VIPER) Lab, Purdue University}{West Lafayette, IN}
\setlength{\itemindent}{.3in}
\item[-] Designed a new loss function for object localization without bounding boxes with $ \ge 90\%$ accuracy.
\item[-] Developed a system based on CNNs and FCNs for plant location and counting from UAV images.
\item[-] Used GANs for data augmentation.
\item[-] Published research at CVPR '19 (Top 1\%), Best Paper Finalist Award **
\end{rSubsection}


\begin{rSubsection}{Research Intern}{May 2017 - Aug 2017}{$\hookrightarrow$ \textbf{Samsung}}{San Jose, CA}
\setlength{\itemindent}{.3in}
\item[-] Developed a new image fidelity metric to evaluate visually lossless compression in Samsung's display pipeline.
\item[-] Result: Higher correlation with subjective evaluation than state-of-the-art metrics.
\end{rSubsection}

\begin{rSubsection}{Research Assistant}{Jan 2015 - Dec 2015}{$\hookrightarrow$ \textbf{Video and Image Processing} (VIPER) Lab, \textbf{Purdue University}}{West Lafayette, IN}
\item[]~~~~~Developed computer vision and image processing techniques for:
\setlength{\itemindent}{.3in}
\item[-] Medical Imaging. Segmention of endocardium in echocardiograms and estimate heart ejection fraction.
\item[-] Visual Surveillance. Counting pedestrian traffic from videos. Improved accuracy by incorporating crowdsourcing.
\end{rSubsection}

\begin{rSubsection}{Web developer and sysadmin}{Jul 2013 - Dec 2013}{$\hookrightarrow$ \textbf{boolino , Telefonica}}{Barcelona, Spain}
\setlength{\itemindent}{.3in}
\item[-] Developed front-end in AngularJS and Bootstrap, and backend in Django and PHP.
\item[-] System administration with Debian Linux
\end{rSubsection}

\vspace{-1pt}

\end{rSection}

%----------------------------------------------------------------------------------------
%	EDUCATION SECTION
%----------------------------------------------------------------------------------------

\vspace{-5pt}
\begin{rSection}{Education}

{\bf PhD, Computer Engineering} ({\normalfont Advisor}: Prof. Edward Delp) -- Purdue University\hfill{\em West Lafayette, IN} \hfill {Jan 2015 - Dec 2018}
%


{\bf BS, Telecommunications Engineering}  -- Polytechnic University of Catalonia  \hfill { \em  Barcelona, Spain} 
 \hfill { Sep 2009 - Dec 2014}

\vspace{-1pt}

\end{rSection}

%----------------------------------------------------------------------------------------
%	PUBLICATIONS SECTION
%----------------------------------------------------------------------------------------

\begin{rSection}{Research Publications (selected)}

\footnotesize
\begin{enumerate}
\setlength{\itemindent}{-.1in}
\item 
\textbf{** Locating objects without bounding boxes} -- \emph{CVPR. Best Paper Finalist Award (Top 1\% of accepted papers)}, June 2019
\item 
\textbf{A machine learning approach to objective image quality evaluation} -- \emph{Society of Information Display - Display Week}, May 2019
\item 
    \textbf{A subpixel-based objective image quality metric [...]} -- \emph{Society of Information Display - Display Week}, May 2018
\item 
\textbf{Counting plants using deep learning} -- \emph{IEEE Global Conference on Signal and Information Processing (GlobalSIP)}, November 2017
\item 
\textbf{Locating crop plant centers from UAV-based RGB imagery} \emph{IEEE International Conference on Computer Vision (ICCV)}, October 2017
\item 
\textbf{Plant leaf segmentation for estimating phenotypic traits} -- \emph{IEEE International Conference on Image Processing (ICIP)}, September 2017
\item 
\textbf{Pill recognition using minimal labeled data} -- \emph{IEEE International Conference on Multimedia Big Data}, April 2017 % \url{https://doi.org/10.1109/BigMM.2017.61}
%\item 
%\textbf{Estimating phenotypic traits from UAV based RGB imagery} -- J. Ribera, F. He, Y. Chen, A. F. Habib, and E. J. Delp, \\
%        \emph{ACM SIGKDD Conference on Knowledge Discovery and Data Mining}, August 2016, San Francisco, CA
\item 
\textbf{Automatic and manual tattoo localization} \emph{IEEE International Conference on Technologies for Homeland Security}, May 2016 % \url{https://doi.org/10.1109/THS.2016.7568950}
%\item 
%\textbf{Characterizing the uncertainty of classification methods and its impact on crowdsourcing} J. Ribera, K. Tahboub, and E. J. Delp, \\
%        \emph{IS\&T/SPIE Electronic Imaging}, February 2015, San Francisco, CA % \url{https://doi.org/10.1117/12.2085415}
\item 
\textbf{An intelligent crowdsourcing system for forensic analysis of surveillance video} -- \emph{IS\&T/SPIE Electronic Imaging}, February 2015
\item 
\textbf{Automated crowd flow estimation enhanced by crowdsourcing} -- \emph{IEEE National Aerospace \& Electronics Conference}, June 2014 %, Dayton, OH % \url{https://doi.org/10.1109/NAECON.2014.7045798}
\end{enumerate}

\vspace{-5pt}

\end{rSection}

%----------------------------------------------------------------------------------------
%	TECHNICAL STRENGTHS SECTION
%----------------------------------------------------------------------------------------

\begin{rSection}{Technical Skills}

\begin{tabular}{ @{} >{\bfseries}l @{\hspace{3ex}} l }
\textbf{Programming Languages} & Python, C, MATLAB, Java, HTML5, Javascript, PHP \\
\textbf{Libraries/Tools} & TensorFlow, PyTorch, scikit-learn, Numpy, OpenCV, Git, AWS, GCP, Docker, Linux systems \\
%\textbf{Languages} & Spanish (native), Catalan (native), French (intermediate) \\
%\textbf{System Administration} & Linux
\end{tabular}

\vspace{-1pt}

\end{rSection}



%----------------------------------------------------------------------------------------
%	VOLUNTEERING SECTION
%----------------------------------------------------------------------------------------

%\begin{rSection}{Volunteering}
%
%\begin{itemize}
%\setlength{\itemindent}{-.2in}
%  \item[-] LinuxUPC student society. Promoted and taught the use of open source software to university students.
%\end{itemize}

%\vspace{-1pt}

%\end{rSection}


\end{document}
