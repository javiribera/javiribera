%%%%%%%%%%%%%%%%%%%%%%%%%%%%%%%%%%%%%%%%%
% Medium Length Professional CV
% LaTeX Template
% Version 2.0 (8/5/13)
%
% This template has been downloaded from:
% http://www.LaTeXTemplates.com
%
% Original author:
% Trey Hunner (http://www.treyhunner.com/)
%
% Important note:
% This template requires the resume.cls file to be in the same directory as the
% .tex file. The resume.cls file provides the resume style used for structuring the
% document.
%
%%%%%%%%%%%%%%%%%%%%%%%%%%%%%%%%%%%%%%%%%

%----------------------------------------------------------------------------------------
%	PACKAGES AND OTHER DOCUMENT CONFIGURATIONS
%----------------------------------------------------------------------------------------

\documentclass{resume} % Use the custom resume.cls style

\usepackage{scrextend}
\usepackage[utf8]{inputenc}
\usepackage[hidelinks]{hyperref}
\usepackage[left=0.5in,top=0.3in,right=0.5in,bottom=0.3in]{geometry} % Document margins
\usepackage{hyperref}

\hypersetup{urlcolor=magenta}

\usepackage{helvet}
\renewcommand{\familydefault}{\sfdefault}
 
\name{Javier Ribera} % Your name
\address{\href{mailto:javi.ribera@gmail.com}{javi.ribera@gmail.com}}
\address{\url{http://ribera.me}}
\begin{document}

% %----------------------------------------------------------------------------------------
% %	OBJECTIVE SECTION
% %----------------------------------------------------------------------------------------
% \vspace{-8pt}
% \begin{rSection}{Objective}
% To obtain a full-time position in Deep Learning applied to Computer Vision/Image Processing, starting January 2019.
% 
% \end{rSection}

%----------------------------------------------------------------------------------------
%	WORK EXPERIENCE SECTION
%----------------------------------------------------------------------------------------

\begin{rSection}{Experience}

\begin{rSubsection}{Sr. Algorithm Engineer}{Jan 2019 - Present}{\textbf{Samsung} Display America Lab}{San Jose, CA}
\setlength{\itemindent}{.3in}
\item[-] Desing and deliver Machine/Deep Learning models to evaluate the image quality of Samsung's new displays.
\end{rSubsection}


\begin{rSubsection}{Research Assistant ({\normalfont Advisor}: Prof. Edward Delp)}{Jan 2016 - Dec 2018}{(ARPA-E Project) Video and Image Processing (VIPER) Lab. Purdue University}{West Lafayette, IN}
\setlength{\itemindent}{.3in}
\item[-] Design a new loss function for object localization without bounding boxes with $ \ge 90\%$ accuracy.
\item[-] Develop a system based on CNNs and FCNs for plant location and counting from UAV images.
\item[-] Employ GANs for data augmentation.
\end{rSubsection}


\begin{rSubsection}{Research Intern}{May - Aug 2017 \& 2018}{\textbf{Samsung} Display America Lab}{San Jose, CA}
\setlength{\itemindent}{.3in}
\item[-] Develop a new image fidelity metric that can model any display and also models the Human Visual System.
\item[-] Evaluate ANNs and CNNs to estimate the perceived quality of an image distorted by a compression algorithm.
\item[-] Result: This metric is better correlated with subjective evaluation than state-of-the-art metrics.
\end{rSubsection}

\begin{rSubsection}{Research Assistant ({\normalfont Advisor}: Prof. Edward Delp)}{Feb 2014 - Dec 2015}{Video and Image Processing (VIPER) Lab. Purdue University}{West Lafayette, IN}
\item[]~~~~~Developed computer vision and image processing techniques for:
\setlength{\itemindent}{.3in}
\item[-] Medical Imaging. Segment endocardium in echocardiograms and estimate heart ejection fraction.
\item[-] Visual Surveillance. Count people from videos. Improved accuracy by incorporating crowdsourcing.
\end{rSubsection}

\vspace{-1pt}

\end{rSection}

%----------------------------------------------------------------------------------------
%	EDUCATION SECTION
%----------------------------------------------------------------------------------------

\begin{rSection}{Education}

{\bf PhD, Electrical and Computer Engineering} \hfill {Jan 2015 - Dec 2018} \\
Purdue University \hfill {\em West Lafayette, IN\\}


{\bf BS, Telecommunications Engineering} \hfill {Sep 2009 - Dec 2014} \\ 
Polytechnic University of Catalonia \hfill {\em Barcelona, Spain}

\vspace{-1pt}

\end{rSection}

%----------------------------------------------------------------------------------------
%	PUBLICATIONS SECTION
%----------------------------------------------------------------------------------------

\begin{rSection}{Publications}

\footnotesize
\begin{enumerate}
\setlength{\itemindent}{-.1in}
\item 
\textbf{``Locating Objects Without Bounding Boxes''} -- Javier Ribera, David Güera, Yuhao Chen, Edward Delp, \\
\emph{Computer Vision and Pattern Recognition (CVPR) Best Paper Finalist Award (Top 1\% of accepted papers)}, June 2019, Long Beach, CA
\item 
\textbf{``A Subpixel-based Objective Image Quality Metric with Application to Visually Lossless Image Compression Evaluation''} -- 
    \\ G. W. Cook, J. Ribera, D. Stolitzka, W. Xiong, \emph{Society of Information Display - Display Week}, May 2018, Los Angeles, CA
\item 
\textbf{``Counting Plants Using Deep Learning''} -- J. Ribera, Y. Chen, C. Boomsma, and E. J. Delp, \\
		\emph{IEEE Global Conference on Signal and Information Processing (GlobalSIP)}, November 2017, Montreal, Canada
\item 
\textbf{``Locating Crop Plant Centers From UAV-based RGB Imagery''} -- Y. Chen, J. Ribera, C. Boomsma, and E. J. Delp, \\
		\emph{IEEE International Conference on Computer Vision Workshops}, October 2017, Venice, Italy
\item 
\textbf{``Plant Leaf Segmentation For Estimating Phenotypic Traits''} -- Y. Chen, J. Ribera, C. Boomsma, and E. J. Delp, \\
		\emph{IEEE International Conference on Image Processing}, September 2017, Beijing, China
\item 
\textbf{``Pill Recognition Using Minimal Labeled Data''} -- Y. Wang, J. Ribera, C. Liu, F. Zhu, and E. J. Delp, \\
		\emph{IEEE International Conference on Multimedia Big Data}, April, 2017, Laguna Hills, CA % \url{https://doi.org/10.1109/BigMM.2017.61}
\item 
\textbf{``Estimating Phenotypic Traits From UAV Based RGB Imagery''} -- J. Ribera, F. He, Y. Chen, A. F. Habib, and E. J. Delp, \\
		\emph{ACM SIGKDD Conference on Knowledge Discovery and Data Mining}, August 2016, San Francisco, CA
\item 
\textbf{``Automatic and Manual Tattoo Localization''} -- J. Kim, H. Li, J. Yue, J. Ribera, L. Huffman, and E. J. Delp, \\
		\emph{IEEE International Conference on Technologies for Homeland Security}, May 2016, Waltham, MA % \url{https://doi.org/10.1109/THS.2016.7568950}
\item 
\textbf{``Characterizing The Uncertainty of Classification Methods and Its Impact on the Performance of Crowdsourcing''} \\-- J. Ribera, K. Tahboub, and E. J. Delp, \emph{IS\&T/SPIE Electronic Imaging}, February 2015, San Francisco, CA % \url{https://doi.org/10.1117/12.2085415}
%\item 
%\textbf{``An Intelligent Crowdsourcing System for Forensic Analysis of Surveillance Video''} -- K. Tahboub, N. Gadgil, J. Ribera, B. Delgado, and E. J. Delp, \emph{IS\&T/SPIE Electronic Imaging}, February 2015, San Francisco, CA. \url{https://doi.org/10.1117/12.2077807}
%\item 
%\textbf{``Automated Crowd Flow Estimation Enhanced by Crowdsourcing''} -- J. Ribera, K. Tahboub, and E. J. Delp, \\
%		\emph{IEEE National Aerospace \& Electronics Conference} June 2014, Dayton, OH % \url{https://doi.org/10.1109/NAECON.2014.7045798}
\end{enumerate}

\vspace{-1pt}

\end{rSection}

%----------------------------------------------------------------------------------------
%	TECHNICAL STRENGTHS SECTION
%----------------------------------------------------------------------------------------

\begin{rSection}{Technical Skills}

\begin{tabular}{ @{} >{\bfseries}l @{\hspace{3ex}} l }
\textbf{Programming} & Python, C, MATLAB, Java, HTML5, Javascript, PHP, Bash \\
\textbf{Libraries/Frameworks} & TensorFlow, PyTorch, Torch, Numpy, OpenCV \\
\textbf{Languages} & Spanish (native), Catalan (native), French (intermediate) \\
\textbf{System Administration} & Linux
\end{tabular}

\vspace{-1pt}

\end{rSection}



%----------------------------------------------------------------------------------------
%	VOLUNTEERING SECTION
%----------------------------------------------------------------------------------------

\begin{rSection}{Volunteering}

\begin{itemize}
\setlength{\itemindent}{-.2in}
  \item[-] Reviewer for IEEE Signal Processing Letters
  \item[-] LinuxUPC student society. Promoted and taught the use of open source software to university students.
\end{itemize}

\vspace{-1pt}

\end{rSection}


\end{document}
